%%%%%%%%%%%%%%%%%%%%%%%%%%%%%%%%%%%%%%%%%%%%%%%%%%%%%%%%%%%%%%%%%%
%%%%%%%%%%%%%%%%%%%%%%%%%%%%%%%%%%%%%%%%%%%%%%%%%%%%%%%%%%%%%%%%%%
%Packages
\documentclass[10pt, a4paper]{article}
\usepackage[UTF8]{ctex}
\usepackage[top=3cm, bottom=4cm, left=3.5cm, right=3.5cm]{geometry}
\usepackage{amsmath,amsthm,amsfonts,amssymb,amscd, fancyhdr, color, comment, graphicx, environ}
\usepackage{float}
\usepackage{mathrsfs}
\usepackage[math-style=ISO]{unicode-math}
\setmathfont{TeX Gyre Termes Math}
\usepackage{lastpage}
\usepackage[dvipsnames]{xcolor}
\usepackage[framemethod=TikZ]{mdframed}
\usepackage{enumerate}
\usepackage[shortlabels]{enumitem}
\usepackage{fancyhdr}
\usepackage{indentfirst}
\usepackage{listings}
\usepackage{sectsty}
\usepackage{thmtools}
\usepackage{shadethm}
\usepackage{hyperref}
\usepackage{CJKfntef}
\usepackage{setspace}
\hypersetup{
    colorlinks=true,
    linkcolor=blue,
    filecolor=magenta,      
    urlcolor=blue,
}
%%%%%%%%%%%%%%%%%%%%%%%%%%%%%%%%%%%%%%%%%%%%%%%%%%%%%%%%%%%%%%%%%%
%%%%%%%%%%%%%%%%%%%%%%%%%%%%%%%%%%%%%%%%%%%%%%%%%%%%%%%%%%%%%%%%%%
%Environment setup
\mdfsetup{skipabove=\topskip,skipbelow=\topskip}
\newrobustcmd\ExampleText{%
An \textit{inhomogeneous linear} differential equation has the form
\begin{align}
L[v ] = f,
\end{align}
where $L$ is a linear differential operator, $v$ is the dependent
variable, and $f$ is a given non−zero function of the independent
variables alone.
}
\mdfdefinestyle{theoremstyle}{%
linecolor=black,linewidth=1pt,%
frametitlerule=true,%
frametitlebackgroundcolor=gray!20,
innertopmargin=\topskip,
}
\mdtheorem[style=theoremstyle]{Problem}{作业题}
\newenvironment{Solution}{}

%%%%%%%%%%%%%%%%%%%%%%%%%%%%%%%%%%%%%%%%%%%%%%%%%%%%%%%%%%%%%%%%%%
%%%%%%%%%%%%%%%%%%%%%%%%%%%%%%%%%%%%%%%%%%%%%%%%%%%%%%%%%%%%%%%%%%
%Fill in the appropriate information below
\newcommand{\norm}[1]{\left\lVert#1\right\rVert}     
\newcommand\course{最优化方法}                         % <-- course name   
%\newcommand\hwnumber{1}                              % <-- homework number
\newcommand\Information{李邹/人工智能一班}          % <-- personal information
%%%%%%%%%%%%%%%%%%%%%%%%%%%%%%%%%%%%%%%%%%%%%%%%%%%%%%%%%%%%%%%%%%
%%%%%%%%%%%%%%%%%%%%%%%%%%%%%%%%%%%%%%%%%%%%%%%%%%%%%%%%%%%%%%%%%%
%Page setup
\pagestyle{fancy}
\headheight 35pt
\lhead{\today}
\rhead{\includegraphics[width=2.5cm]{lzu-logo.png}}
\lfoot{}
\pagenumbering{arabic}
\cfoot{\small\thepage}
\rfoot{}
\headsep 1.2em
\renewcommand{\baselinestretch}{1.25}
%%%%%%%%%%%%%%%%%%%%%%%%%%%%%%%%%%%%%%%%%%%%%%%%%%%%%%%%%%%%%%%%%%
%%%%%%%%%%%%%%%%%%%%%%%%%%%%%%%%%%%%%%%%%%%%%%%%%%%%%%%%%%%%%%%%%%
%Add new commands here
\renewcommand{\labelenumi}{\alph{enumi})}
\newcommand{\Z}{\mathbb Z}
\newcommand{\R}{\mathbb R}
\newcommand{\Q}{\mathbb Q}
\newcommand{\NN}{\mathbb N}
\DeclareMathOperator{\Mod}{Mod} 
\renewcommand\lstlistingname{Algorithm}
\renewcommand\lstlistlistingname{Algorithms}
\def\lstlistingautorefname{Alg.}
%%%%%%%%%%%%%%%%%%%%%%%%%%%%%%%%%%%%%%%%%%%%%%%%%%%%%%%%%%%%%%%%%%
%%%%%%%%%%%%%%%%%%%%%%%%%%%%%%%%%%%%%%%%%%%%%%%%%%%%%%%%%%%%%%%%%%
%Begin now!



\begin{document}

\begin{titlepage}
    \begin{center}
        \vspace*{3.5cm}
            
        \Huge
        \textbf{作业7}
            
        \vspace{2cm}
        \LARGE
        李邹
            
        \vspace{0.1cm}
        \Large
        人工智能一班(2020级)                      % <-- author
        
            
        \vfill
        
        \course \ 课程作业
            
        \vspace{1cm}
            
        \includegraphics[width=0.4\textwidth]{lzu-logo.png}
        \\
        
        \Large
        
        \today
            
    \end{center}
\end{titlepage}

%%%%%%%%%%%%%%%%%%%%%%%%%%%%%%%%%%%%%%%%%%%%%%%%%%%%%%%%%%%%%%%%%%
%%%%%%%%%%%%%%%%%%%%%%%%%%%%%%%%%%%%%%%%%%%%%%%%%%%%%%%%%%%%%%%%%%
%Start the assignment now

%%%%%%%%%%%%%%%%%%%%%%%%%%%%%%%%%%%%%%%%%%%%%%%%%%%%%%%%%%%%%%%%%%
%New problem
\newpage
\begin{Problem}
\textbf{拟凸性的一阶条件。}\;证明\,§3.4.3\,中给出的判断拟凸性的一阶条件:可微函数$f:\mathrm {R} ^{n}\longrightarrow \mathrm {R}  $,其定义域$\textbf{dom}\,f$是凸集,则函数$f$是拟凸函数的充要条件是,对任意$x,y\in \textbf{dom}\,f$有
\[f(y) \leq f(x) \Longrightarrow \nabla f(x)^{T}(y-x) \leq 0\]
\end{Problem}
    
\begin{Solution}
\textbf{证明}
\end{Solution}
%Complete the assignment now
\textbf{首先证明:}$f$ 是拟凸函数能推出:若 $f(y) \leq f(x) \Rightarrow \nabla f^{T}(x)(y-x) \leq 0$

因为 $f(x)$ 为拟凸函数,所以对于$\forall x, y \in \operatorname{dom} f, 0 \leq \theta \leq 1$,都有
\[\max \{f(x), f(y)\} \geq f(\theta x+(1-\theta) y)\]

设 $f(y) \leq f(x)$,从而有
\begin{align}
    &f(x) \geq f(\theta x+(1-\theta) y) \notag \\
    \Leftrightarrow &f(\theta x+(1-\theta) y)-f(x) \leq 0 \notag \\
    \Leftrightarrow &f(\theta x+(1-\theta) y)-f(\theta x+(1-\theta) x) \leq 0 \notag \\
    \Leftrightarrow &\frac{f(\theta x+(1-\theta) y)-f(\theta x+(1-\theta) x)}{(1-\theta)(y-x)}(1-\theta)(y-x) \leq 0 
\end{align}

当$\theta \to 0 $时,(1)式则化为
\[f^{\prime}(x)(y-x) \leq 0\]

\textbf{其次证明:}若 $\forall x, y \in \operatorname{dom} f, f(y) \leq f(x) \Rightarrow \nabla f^{T}(x)(y-x) \leq 0$ 能推出 $f$ 是拟凸函数

不妨设$z=\theta x+(1-\theta) y\,,\,0<\theta<1$,则等价于证明$f(x) \geq f(z)$成立。
\vspace*{1em}

$f(x)$与$f(z)$的大小关系有以下两种情况:
\begin{enumerate}[1)]
    \item \textbf{$f(x)>f(z)$}\\显然符合$f(x) \geq f(z)$.
    \item \textbf{$f(x) \leq f(z)$}\\又因为有$f(y) \leq f(x)$,从而有
    \begin{align}
        &f^{\prime}(z)(x-z) \leq 0, f^{\prime}(z)(y-z) \leq 0 \notag \\
        \Rightarrow &f^{\prime}(z)(1-\theta)(x-y) \leq 0, f^{\prime}(z) \theta(y-x) \leq 0 \notag \\
        \Rightarrow &f^{\prime}(z)=0 \notag
    \end{align}

    可在 $[z, x]$ 的区间上不断利用上述结论,因为 $f$ 为连续函数, 最终必然有 $f(z)=f(x)$.
\end{enumerate}

因此,$f(x) \leq f(x)$恒成立,故其等价命题也成立。
\vspace*{1em}

\textbf{综上,原命题得证。}
\hfill $\Box$
\vspace*{3em}

\begin{Problem}
设函数$f:\mathrm {R} ^{n}\longrightarrow \mathrm {R}  $可微,定义域$\textbf{dom}\,f$是凸集,对任意$x\in \textbf{dom}\,f$,$f(x)>0$。证明函数$f$是对数-凹函数的充要条件是,对于任意$x,y\in \textbf{dom}\,f$,下式成立
\[\frac{f(y)}{f(x)} \leq \exp \left(\frac{\nabla f(x)^{T}(y-x)}{f(x)}\right)\]
\end{Problem}

\begin{Solution}
\textbf{证明}
\end{Solution}
%Complete the assignment now
\textbf{先证必要性:}令$g(x)=\ln_{}{f(x)} $,不等式两边同取自然对数,则有:
\begin{align}
    &\frac{f(y)}{f(x)} \leq \exp \left(\frac{\nabla f(x)^{T}(y-x)}{f(x)}\right) \notag \\
    \Longrightarrow &\ln_{}{f(y)} -\ln_{}{f(x)} \le \frac{\nabla f(x)^{T}(y-x)}{f(x)} \notag \\
    \Longrightarrow &g(y)-g(x)\le \nabla g(x)^{T}(y-x) \notag
\end{align}

由一阶条件可推知$g$是凹函数,因此$f$是对数-凹函数。
\vspace*{1em}

\textbf{再证充分性:}因为$f$是对数-凹函数,所以$g$是凹函数,则有:
\begin{align}
    &g(y)-g(x)\le \nabla g(x)^{T}(y-x) \notag \\
    \Longrightarrow& \ln_{}{f(y)} -\ln_{}{f(x)} \le \frac{\nabla f(x)^{T}(y-x)}{f(x)} \notag \\
    \Longrightarrow& \frac{f(y)}{f(x)} \leq \exp \left(\frac{\nabla f(x)^{T}(y-x)}{f(x)}\right) \notag
\end{align}

\textbf{综上,原命题得证。}
\hfill $\Box$
\vspace*{3em}

\begin{Problem}
证明如果函数$f:\mathrm {R} ^{n}\longrightarrow \mathrm {R}  $是对数-凹函数且$a\ge 0$,那么函数$g=f-a$是对数-凹函数,其定义域为$\textbf{dom}\,g=\{x \in \textbf{dom}\,f \mid f(x)>a\}$。
\end{Problem}
    
\begin{Solution}
\textbf{证明}
\end{Solution}
%Complete the assignment now
由题有,$f$是对数-凹函数,则有:
\begin{align}
    f(\theta x+(1-\theta) y) \geq f(x)^{\theta} f(y)^{1-\theta}
\end{align}

将(2)式代入$g=f-a$中,利用霍尔德不等式,则有:
\begin{align}
    f(\theta x+(1-\theta) y) -a &\geq f(x)^{\theta} f(y)^{1-\theta}-a  = f(x)^{\theta} f(y)^{1-\theta}- a^{\theta} a^{1-\theta} \notag \\
    &\geq (f(x)-a)^{\theta} (f(y)-a)^{1-\theta} \notag
\end{align}

所以函数$g$是对数-凹函数,定义域为$\textbf{dom} g=\{x \in \textbf{dom} f \mid f(x)>a\}$.
\hfill $\Box$
\vspace*{3em}

\begin{Problem}
证明下列函数是对数-凹函数
\\
\textbf{调和平均函数}
\[f(x)=\frac{1}{1 / x_{1}+\cdots+1 / x_{n}}, \quad \textbf{dom} f=\mathbf{R}_{++}^{n}\]
\end{Problem}
        
\begin{Solution}
\textbf{证明}
\end{Solution}
%Complete the assignment now
对函数$f$求偏导,结果如下:
\begin{align}
    \frac{\partial f}{\partial x_{i}} & = \frac{1}{\left(a x_{i}+1\right)^{2}}>0, \quad a = \sum_{n \neq i} \frac{1}{x_{n}} \notag \\
    \frac{\partial^{2} f}{\partial x_{i}^{2}} & = -\frac{2 a}{\left(a x_{i}+1\right)^{3}}<0, \quad a = \sum_{n \neq i} \frac{1}{x_{n}} \notag \\
    \frac{\partial^{2} f}{\partial x_{i} \partial x_{j}} & = \frac{2 x_{i} x_{j}}{\left(b x_{i} x_{j}+x_{i}+x_{j}\right)^{3}}>0, \quad b = \sum_{n \neq i, j} \frac{1}{x_{n}} \notag
\end{align}

显然有$\nabla^{2} f(x) \preceq 0, \nabla f(x) \succeq 0$,从而可以得到
\begin{align}
    &f(x) \nabla^{2} f(x) \preceq 0, \quad f(x) \nabla f(x)^{T} \succeq 0 \notag \\
    \Longrightarrow &f(x) \nabla^{2} f(x) \preceq \nabla f(x) \nabla f(x)^{T} \notag
\end{align}

所以,函数$f$是对数-凹函数。
\hfill $\Box$
\end{document}

%%%%%%%%%%%%%%%%%%%%%%%%%%%%%%%%%%%%%%%%%%%%%%%%%%%%%%%%%%%%%%%%%%
%%%%%%%%%%%%%%%%%%%%%%%%%%%%%%%%%%%%%%%%%%%%%%%%%%%%%%%%%%%%%%%%%%
