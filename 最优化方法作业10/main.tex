%%%%%%%%%%%%%%%%%%%%%%%%%%%%%%%%%%%%%%%%%%%%%%%%%%%%%%%%%%%%%%%%%%
%%%%%%%%%%%%%%%%%%%%%%%%%%%%%%%%%%%%%%%%%%%%%%%%%%%%%%%%%%%%%%%%%%
%Packages
\documentclass[10pt, a4paper]{article}
\usepackage[UTF8]{ctex}
\usepackage[top=3cm, bottom=4cm, left=3.5cm, right=3.5cm]{geometry}
\usepackage{amsmath,amsthm,amsfonts,amssymb,amscd, fancyhdr, color, comment, graphicx, environ}
\usepackage{float}
\usepackage{mathrsfs}
\usepackage[math-style=ISO]{unicode-math}
\setmathfont{TeX Gyre Termes Math}
\usepackage{lastpage}
\usepackage[dvipsnames]{xcolor}
\usepackage[framemethod=TikZ]{mdframed}
\usepackage{enumerate}
\usepackage[shortlabels]{enumitem}
\usepackage{fancyhdr}
\usepackage{indentfirst}
\usepackage{listings}
\usepackage{sectsty}
\usepackage{thmtools}
\usepackage{shadethm}
\usepackage{hyperref}
\usepackage{CJKfntef}
\usepackage{setspace}
\hypersetup{
    colorlinks=true,
    linkcolor=blue,
    filecolor=magenta,      
    urlcolor=blue,
}
%%%%%%%%%%%%%%%%%%%%%%%%%%%%%%%%%%%%%%%%%%%%%%%%%%%%%%%%%%%%%%%%%%
%%%%%%%%%%%%%%%%%%%%%%%%%%%%%%%%%%%%%%%%%%%%%%%%%%%%%%%%%%%%%%%%%%
%Environment setup
\mdfsetup{skipabove=\topskip,skipbelow=\topskip}
\newrobustcmd\ExampleText{%
An \textit{inhomogeneous linear} differential equation has the form
\begin{align}
L[v ] = f,
\end{align}
where $L$ is a linear differential operator, $v$ is the dependent
variable, and $f$ is a given non−zero function of the independent
variables alone.
}
\mdfdefinestyle{theoremstyle}{%
linecolor=black,linewidth=1pt,%
frametitlerule=true,%
frametitlebackgroundcolor=gray!20,
innertopmargin=\topskip,
}
\mdtheorem[style=theoremstyle]{Problem}{作业题}
\newenvironment{Solution}{}

%%%%%%%%%%%%%%%%%%%%%%%%%%%%%%%%%%%%%%%%%%%%%%%%%%%%%%%%%%%%%%%%%%
%%%%%%%%%%%%%%%%%%%%%%%%%%%%%%%%%%%%%%%%%%%%%%%%%%%%%%%%%%%%%%%%%%
%Fill in the appropriate information below
\newcommand{\norm}[1]{\left\lVert#1\right\rVert}     
\newcommand\course{最优化方法}                         % <-- course name   
%\newcommand\hwnumber{1}                              % <-- homework number
\newcommand\Information{李邹/人工智能一班}          % <-- personal information
%%%%%%%%%%%%%%%%%%%%%%%%%%%%%%%%%%%%%%%%%%%%%%%%%%%%%%%%%%%%%%%%%%
%%%%%%%%%%%%%%%%%%%%%%%%%%%%%%%%%%%%%%%%%%%%%%%%%%%%%%%%%%%%%%%%%%
%Page setup
\pagestyle{fancy}
\headheight 35pt
\lhead{\today}
\rhead{\includegraphics[width=2.5cm]{lzu-logo.png}}
\lfoot{}
\pagenumbering{arabic}
\cfoot{\small\thepage}
\rfoot{}
\headsep 1.2em
\renewcommand{\baselinestretch}{1.25}
%%%%%%%%%%%%%%%%%%%%%%%%%%%%%%%%%%%%%%%%%%%%%%%%%%%%%%%%%%%%%%%%%%
%%%%%%%%%%%%%%%%%%%%%%%%%%%%%%%%%%%%%%%%%%%%%%%%%%%%%%%%%%%%%%%%%%
%Add new commands here
\renewcommand{\labelenumi}{\alph{enumi})}
\newcommand{\Z}{\mathbb Z}
\newcommand{\R}{\mathbb R}
\newcommand{\Q}{\mathbb Q}
\newcommand{\NN}{\mathbb N}
\DeclareMathOperator{\Mod}{Mod} 
\renewcommand\lstlistingname{Algorithm}
\renewcommand\lstlistlistingname{Algorithms}
\def\lstlistingautorefname{Alg.}
%%%%%%%%%%%%%%%%%%%%%%%%%%%%%%%%%%%%%%%%%%%%%%%%%%%%%%%%%%%%%%%%%%
%%%%%%%%%%%%%%%%%%%%%%%%%%%%%%%%%%%%%%%%%%%%%%%%%%%%%%%%%%%%%%%%%%
%Begin now!



\begin{document}

\begin{titlepage}
    \begin{center}
        \vspace*{3.5cm}
            
        \Huge
        \textbf{作业10}
            
        \vspace{2cm}
        \LARGE
        李邹
            
        \vspace{0.1cm}
        \Large
        人工智能一班(2020级)                      % <-- author
        
            
        \vfill
        
        \course \ 课程作业
            
        \vspace{1cm}
            
        \includegraphics[width=0.4\textwidth]{lzu-logo.png}
        \\
        
        \Large
        
        \today
            
    \end{center}
\end{titlepage}

%%%%%%%%%%%%%%%%%%%%%%%%%%%%%%%%%%%%%%%%%%%%%%%%%%%%%%%%%%%%%%%%%%
%%%%%%%%%%%%%%%%%%%%%%%%%%%%%%%%%%%%%%%%%%%%%%%%%%%%%%%%%%%%%%%%%%
%Start the assignment now

%%%%%%%%%%%%%%%%%%%%%%%%%%%%%%%%%%%%%%%%%%%%%%%%%%%%%%%%%%%%%%%%%%
%New problem
\newpage
\begin{Problem}
加热到$T$(高于环境温度)的流体在长度固定、截面半径为$r$的圆形管道种流动。管道外有一层厚度$w \ll r$的绝热涂层以减少透过管壁的热损失。
这个问题的设计变量为$T,r$和$w$。\\
热量损失(近似)正比于$Tr/w$,所以在固定的使用期限内,由热量损失带来的能量损失由$\alpha_{1} Tr/w$给出。具有固定管壁厚度的管道的成本近似正比于总材料,由$\alpha_{2} r$给出。涂层的成本也近似正比于总的涂料,
即$\alpha_{3} rw$(利用$w \ll r$)。总成本是这三种成本之和。\\
管道中流过的热量完全取决于流体的流量(流体具有固定的速度),由$\alpha_{4}Tr^{2}$给出。如同变量$T,r$和$w$一样,常数$\alpha_{i}$为正。\\
现在的问题是:在总成本限制$C_{max}$和约束
$$
T_{\min } \leq T \leq T_{\max }, \quad r_{\min } \leq r \leq r_{\max }, \quad w_{\min } \leq w \leq w_{\max }, \quad w \leq 0.1 r
$$
下,极大化管道运输的总热量。将这个问题表示为一个几何规划(GP)。
\end{Problem}
    
\begin{Solution}
\textbf{解答}
\end{Solution}
%Complete the assignment now
根据题意,我们可以先写出该优化问题的一般形式。

题意要求极大化管道运输的总热量,因此有
\[\operatorname{maximize} \; \alpha_{4} T r^{2}\]

除直接给定的约束外,题中还给出了总成本限制$C_{max}$,因此有
\[\alpha_{1} T w^{-1}+\alpha_{2} r+\alpha_{3} r w \leq C_{\max }\]

从而,可以得到一般形式
$$
\begin{array}{ll}
\operatorname{maximize} & \alpha_{4} T r^{2} \\
\text {subject to } & \alpha_{1} T w^{-1}+\alpha_{2} r+\alpha_{3} r w \leq C_{\max } \\
& T_{\min } \leq T \leq T_{\max } \\
& r_{\min } \leq r \leq r_{\max } \\
& w_{\min } \leq w \leq w_{\max } \\
& w \leq 0.1 r
\end{array}
$$

将其转换为等价的几何规划(GP)
$$
\begin{array}{ll}
\operatorname{minimize} & \left(1 / \alpha_{4}\right) T^{-1} r^{-2} \\
\text {subject to } & \left(\alpha_{1} / C_{\max }\right) T w^{-1}+\left(\alpha_{2} / C_{\max }\right) r+\left(\alpha_{3} / C_{\max }\right) r w \leq 1 \\
& \left(1 / T_{\max }\right) T \leq 1, \quad T_{\min } T^{-1} \leq 1 \\
& \left(1 / r_{\max }\right) r \leq 1, \quad r_{\min } r^{-1} \leq 1 \\
& \left(1 / w_{\max }\right) w \leq 1, \quad w_{\min } w^{-1} \leq 1 \\
& 10 w r^{-1} \leq 1 .
\end{array}
$$
\hfill $\Box$
\vspace*{3em}

\begin{Problem}
\textbf{最佳梁设计问题的递归形式。}\enspace 证明GP(4.46)等价于GP
$$
\begin{array}{ll}
\operatorname{minimize} & \sum_{i=1}^{N} w_{i} h_{i} \\
\text {subject to } & w_{i} / w_{\max } \leq 1, \quad w_{\min } / w_{i} \leq 1, \quad i=1, \ldots, N \\
& h_{i} / h_{\max } \leq 1, \quad h_{\min } / h_{i} \leq 1, \quad i=1, \ldots, N \\
& h_{i} /\left(w_{i} S_{\max }\right) \leq 1 \quad i=1, \ldots, N \\
& 6 i F /\left(\sigma_{\max } w_{i} h_{i}^{2}\right) \leq 1, \quad i=1, \ldots, N \\
& (2 i-1) d_{i} / v_{i}+v_{i+1} / v_{i} \leq 1, \quad i=1, \ldots, N \\
& (i-1 / 3) d_{i} / y_{i}+v_{i+1} / y_{i}+y_{i+1} / y_{i} \leq 1, \quad i=1, \ldots, N \\
& y_{1} / y_{\max } \leq 1 \\
& E w_{i} h_{i}^{3} d_{i} /(6 F)=1, \quad i=1, \ldots, N .
\end{array}
$$
其变量为$w_{i}, h_{i}, v_{i}, d_{i}, y_{i}$ for $i=1, \ldots, N$。
\end{Problem}

\begin{Solution}
\textbf{证明}
\end{Solution}
%Complete the assignment now
GP(4.46)中给出了如下的优化问题
$$
\begin{array}{ll}
\operatorname{minimize} & \sum_{i=1}^{N} w_{i} h_{i} \\
\text { subject to } & w_{\min } \leq w_{i} \leq w_{\max }, i=1, \ldots, N \\
& h_{\min } \leq h_{i} \leq h_{\max }, i=1, \ldots, N \\
& S_{\min } \leq h_{i} / w_{i} \leq S_{\max }, i=1, \ldots, N \\
& 6 i F / w_{i} h_{i}^{2} \leq \sigma_{\max }, i=1, \ldots, N \\
& 12(2 i-1) F / E w_{i} h_{i}^{3}+v_{i+1}=v_{i}, i=1, \ldots, N \\
& 6(i-1 / 3) F / E w_{i} h_{i}^{3}+v_{i+1}+y_{i+1}=y_{i}, i=1, \ldots, N \\
& y_{1} \leq y_{\max }
\end{array}
$$

对于其中的两个等式
\[v_{i}=12(i-1 / 2) \frac{F}{E w_{i} h_{i}^{3}}+v_{i+1}, \quad y_{i}=6(i-1 / 3) \frac{F}{E w_{i} h_{i}^{3}}+v_{i+1}+y_{i+1}\]

结合该问题的实际物理意义,即让悬臂梁保持平衡状态,因此可以得到
\[v_{i} \geq 12(i-1 / 2) \frac{F}{E w_{i} h_{i}^{3}}+v_{i+1}, \quad y_{i} \geq 6(i-1 / 3) \frac{F}{E w_{i} h_{i}^{3}}+v_{i+1}+y_{i+1}\]

再将GP(4.46)中的其他约束化为几何规划的形式,并令
\[d_{i}=6 F /\left(E w_{i} h_{i}^{3}\right), \quad i=1, \ldots, N\]

显然和题目中的优化问题是等价的。

\hfill $\Box$
\vspace*{3em}

\begin{Problem}
将下面的问题表示为凸优化问题。
\begin{enumerate}[(a)]
    \item 极小化$\max \{p(x), q(x)\} $,其中$p$和$q$为正项式。
    \item 极小化$\exp (p(x))+\exp (q(x))$,其中$p$和$q$为正项式。
    \item 在$r(x)>q(x)$的约束下极小化$p(x) /(r(x)-q(x))$,其中$p$和$q$为正项式,$r$为单项式。
\end{enumerate}
\end{Problem}
    
\begin{Solution}
\textbf{解答}
\end{Solution}
%Complete the assignment now
\textbf{以下三问的核心思路,都是通过对数函数,将包含正项式的目标函数与约束函数进行转化,以得到凸形式的几何规划。}
\begin{enumerate}[(a)]
    \item 因为$p(x),q(x)$是正项式,因此有
    $$
\begin{aligned}
&p(x)=\sum_{k=1}^{N} c_{k} x_{1}^{a_{1 k}} x_{2}^{a_{2 k}} \cdots x_{n}^{a_{n k}} \\
&q(x)=\sum_{k=1}^{M} f_{k} x_{1}^{d_{1 k}} x_{2}^{d_{2 k}} \cdots x_{n}^{d_{n k}}
\end{aligned}
$$

令$y_{i}=\log x_{i}$,则有
$$
\begin{aligned}
&p(x)=\sum_{k=1}^{N} e^{a_{k}^{T} y+b_{k}} \\
&q(x)=\sum_{k=1}^{M} e^{d_{k}^{T} y+e_{k}}
\end{aligned}
$$

那么可以得到相应的几何规划
$$
\begin{aligned}
    \min & \;t\\
    \text { s.t. } &\sum_{k=1}^{N} e^{a_{k}^{T} y+b_{k}} t^{-1} \leq 1\\
    &\sum_{k=1}^{M} e^{d_{k}^{T} y+e_{k}} t^{-1} \leq 1
\end{aligned}
$$

使用对数函数进行转换,代回原变元,可以到到最终形式
$$
\begin{array}{ll}
\operatorname{minimize} & t \\
\text {subject to } & p(x) / t \leq 1, \quad q(x) / t \leq 1
\end{array}
$$
    \item 和(a)中的步骤类似,我们可以转换为如下的几何规划
    $$
    \begin{aligned}
    \min & \;\exp \left(t_{1}\right)+\exp \left(t_{2}\right)\\
    \text { s.t. } & \sum_{k=1}^{N} e^{a_{k}^{T} y+b_{k}} t_{1}^{-1} \leq 1 \\
    & \sum_{k=1}^{M} e^{d_{k}^{T} y+e_{k}} t_{2}^{-1} \leq 1
    \end{aligned}
    $$

    使用对数函数进行转换,可以得到
    $$
    \begin{array}{ll}
    \min & \log \left(\exp \left(t_{1}\right)+\exp \left(t_{2}\right)\right)\\
    \text { s.t. } & \log \left(\sum_{k=1}^{N} e^{a_{k}^{T} y+b_{k}}\right)+\log \left(t_{1}^{-1}\right) \leq 0 \\
    & \log \left(\sum_{k=1}^{M} e^{d_{k}^{T} y+e_{k}}\right)+\log \left(t_{2}^{-1}\right) \leq 0
    \end{array}
    $$

    代回原变元,可以到到最终形式
    $$
    \begin{array}{ll}
    \operatorname{minimize} & \exp \left(t_{1}\right)+\exp \left(t_{2}\right) \\
    \text {subject to } & p(x) \leq t_{1}, \quad q(x) \leq t_{2}
    \end{array}
    $$
    \item 同(a)(b)过程一致,只需注意$r(x)$是单项式即可。
    
    可以得到使用对数函数进行转换后的几何规划为
    $$
    \begin{array}{ll}
    \min & \log (t)\\
    \text { s.t. } & \log \left(\sum_{k=1}^{N} e^{a_{k}^{T} y+b_{k}} t^{-1} /\left(e^{h^{T} y+i}-\sum_{k=1}^{M} e^{d_{k}^{T} y+e_{k}}\right)\right) \leq 0 \\
    & \log \left(\sum_{k=1}^{M} e^{d_{k}^{T} y+e_{k}} / e^{h^{T} y+i}\right) \leq 0
    \end{array}
    $$

    代回原变元,可以到到最终形式
    $$
    \begin{array}{ll}
    \operatorname{minimize} & t \\
    \text {subject to } & p(x) \leq t(r(x)-q(x))
    \end{array}
    $$
\end{enumerate}

\end{document}

%%%%%%%%%%%%%%%%%%%%%%%%%%%%%%%%%%%%%%%%%%%%%%%%%%%%%%%%%%%%%%%%%%
%%%%%%%%%%%%%%%%%%%%%%%%%%%%%%%%%%%%%%%%%%%%%%%%%%%%%%%%%%%%%%%%%%
