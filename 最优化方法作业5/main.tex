%%%%%%%%%%%%%%%%%%%%%%%%%%%%%%%%%%%%%%%%%%%%%%%%%%%%%%%%%%%%%%%%%%
%%%%%%%%%%%%%%%%%%%%%%%%%%%%%%%%%%%%%%%%%%%%%%%%%%%%%%%%%%%%%%%%%%
%Packages
\documentclass[10pt, a4paper]{article}
\usepackage[UTF8]{ctex}
\usepackage[top=3cm, bottom=4cm, left=3.5cm, right=3.5cm]{geometry}
\usepackage{amsmath,amsthm,amsfonts,amssymb,amscd, fancyhdr, color, comment, graphicx, environ}
\usepackage{float}
\usepackage{mathrsfs}
\usepackage[math-style=ISO]{unicode-math}
\setmathfont{TeX Gyre Termes Math}
\usepackage{lastpage}
\usepackage[dvipsnames]{xcolor}
\usepackage[framemethod=TikZ]{mdframed}
\usepackage{enumerate}
\usepackage[shortlabels]{enumitem}
\usepackage{fancyhdr}
\usepackage{indentfirst}
\usepackage{listings}
\usepackage{sectsty}
\usepackage{thmtools}
\usepackage{shadethm}
\usepackage{hyperref}
\usepackage{CJKfntef}
\usepackage{setspace}
\hypersetup{
    colorlinks=true,
    linkcolor=blue,
    filecolor=magenta,      
    urlcolor=blue,
}
%%%%%%%%%%%%%%%%%%%%%%%%%%%%%%%%%%%%%%%%%%%%%%%%%%%%%%%%%%%%%%%%%%
%%%%%%%%%%%%%%%%%%%%%%%%%%%%%%%%%%%%%%%%%%%%%%%%%%%%%%%%%%%%%%%%%%
%Environment setup
\mdfsetup{skipabove=\topskip,skipbelow=\topskip}
\newrobustcmd\ExampleText{%
An \textit{inhomogeneous linear} differential equation has the form
\begin{align}
L[v ] = f,
\end{align}
where $L$ is a linear differential operator, $v$ is the dependent
variable, and $f$ is a given non−zero function of the independent
variables alone.
}
\mdfdefinestyle{theoremstyle}{%
linecolor=black,linewidth=1pt,%
frametitlerule=true,%
frametitlebackgroundcolor=gray!20,
innertopmargin=\topskip,
}
\mdtheorem[style=theoremstyle]{Problem}{作业题}
\newenvironment{Solution}{\textbf{证明}\\}

%%%%%%%%%%%%%%%%%%%%%%%%%%%%%%%%%%%%%%%%%%%%%%%%%%%%%%%%%%%%%%%%%%
%%%%%%%%%%%%%%%%%%%%%%%%%%%%%%%%%%%%%%%%%%%%%%%%%%%%%%%%%%%%%%%%%%
%Fill in the appropriate information below
\newcommand{\norm}[1]{\left\lVert#1\right\rVert}     
\newcommand\course{最优化方法}                         % <-- course name   
%\newcommand\hwnumber{1}                              % <-- homework number
\newcommand\Information{李邹/人工智能一班}          % <-- personal information
%%%%%%%%%%%%%%%%%%%%%%%%%%%%%%%%%%%%%%%%%%%%%%%%%%%%%%%%%%%%%%%%%%
%%%%%%%%%%%%%%%%%%%%%%%%%%%%%%%%%%%%%%%%%%%%%%%%%%%%%%%%%%%%%%%%%%
%Page setup
\pagestyle{fancy}
\headheight 35pt
\lhead{\today}
\rhead{\includegraphics[width=2.5cm]{lzu-logo.png}}
\lfoot{}
\pagenumbering{arabic}
\cfoot{\small\thepage}
\rfoot{}
\headsep 1.2em
\renewcommand{\baselinestretch}{1.25}
%%%%%%%%%%%%%%%%%%%%%%%%%%%%%%%%%%%%%%%%%%%%%%%%%%%%%%%%%%%%%%%%%%
%%%%%%%%%%%%%%%%%%%%%%%%%%%%%%%%%%%%%%%%%%%%%%%%%%%%%%%%%%%%%%%%%%
%Add new commands here
\renewcommand{\labelenumi}{\alph{enumi})}
\newcommand{\Z}{\mathbb Z}
\newcommand{\R}{\mathbb R}
\newcommand{\Q}{\mathbb Q}
\newcommand{\NN}{\mathbb N}
\DeclareMathOperator{\Mod}{Mod} 
\renewcommand\lstlistingname{Algorithm}
\renewcommand\lstlistlistingname{Algorithms}
\def\lstlistingautorefname{Alg.}
%%%%%%%%%%%%%%%%%%%%%%%%%%%%%%%%%%%%%%%%%%%%%%%%%%%%%%%%%%%%%%%%%%
%%%%%%%%%%%%%%%%%%%%%%%%%%%%%%%%%%%%%%%%%%%%%%%%%%%%%%%%%%%%%%%%%%
%Begin now!



\begin{document}

\begin{titlepage}
    \begin{center}
        \vspace*{3.5cm}
            
        \Huge
        \textbf{作业5}
            
        \vspace{2cm}
        \LARGE
        李邹
            
        \vspace{0.1cm}
        \Large
        人工智能一班(2020级)                      % <-- author
        
            
        \vfill
        
        \course \ 课程作业
            
        \vspace{1cm}
            
        \includegraphics[width=0.4\textwidth]{lzu-logo.png}
        \\
        
        \Large
        
        \today
            
    \end{center}
\end{titlepage}

%%%%%%%%%%%%%%%%%%%%%%%%%%%%%%%%%%%%%%%%%%%%%%%%%%%%%%%%%%%%%%%%%%
%%%%%%%%%%%%%%%%%%%%%%%%%%%%%%%%%%%%%%%%%%%%%%%%%%%%%%%%%%%%%%%%%%
%Start the assignment now

%%%%%%%%%%%%%%%%%%%%%%%%%%%%%%%%%%%%%%%%%%%%%%%%%%%%%%%%%%%%%%%%%%
%New problem
\newpage
\begin{Problem}
证明:反透视函数是保凸的。
\\\\\\
test
\end{Problem}
    
\begin{Solution}

\end{Solution}
    

%%%%%%%%%%%%%%%%%%%%%%%%%%%%%%%%%%%%%%%%%%%%%%%%%%%%%%%%%%%%%%%%%%
%Complete the assignment now
题目即要求证明,如果$C\subseteq \mathbb{R} ^{n} $为凸集,那么\[P^{-1} (C)=\left \{ (x,t)\in \mathbb{R}^{n+1}\mid \frac{x}{t}\in C\;,\;t>0 \right \} \]

也是凸集。\\

在$P^{-1} (C)$中任取两点$(x,t)$和$(y,s)$,且有$\theta \in \left [ 0,1 \right ] $,即证明:\[(\theta x+(1-\theta )y\;,\;\theta t+(1-\theta )s)\in P^{-1} (C)\]

即证明:\[\frac{\theta x+(1-\theta )y}{\theta t+(1-\theta )s} \in C\]

可以得到,
\begin{equation*}
    \begin{aligned}
        & \frac{\theta x+(1-\theta )y}{\theta t+(1-\theta )s} \\
        \iff & \frac{\theta t}{\theta t+(1-\theta )s}\cdot \frac{x}{t}  +\frac{(1-\theta)s}{\theta t+(1-\theta )s}\cdot \frac{y}{s}  \\
        \iff & \mu \frac{x}{t} +(1-\mu )\frac{y}{s} \in C\ \\
    \end{aligned}
  \end{equation*}

故原命题得证。

\hfill $\Box$ 

\begin{Problem}
证明零阶条件:$x \in \mathbf{dom} \ f$,$v \in \mathbb{R}^n$,$t \in \mathbb{R}$,$x+tv \in \mathbf{dom} \ f$, 函数$f$是凸的,当且仅当对于任意$x \in \mathbf{dom} \ f$和任意向量$v$,函数$g(t)= f(x+tv)$是凸的。
\end{Problem}

\begin{Solution}

\end{Solution}

\textbf{先证:}$f$是凸的$\Longrightarrow$ $g$是凸的

取$\forall t_{1} ,t_{2} \in \mathbf{dom}\ f$,并且$\alpha  \in \left [ 0,1 \right ]$.

则有,
\begin{equation*}
    \begin{aligned}
        g\left(\alpha t_{1}+(1-\alpha) t_{2}\right) &=f\left(x+\left(\alpha t_{1}+(1-\alpha) t_{2}\right) v\right) \\
        &=f\left(\alpha x+\alpha t_{1} v+(1-\alpha) x+(1-\alpha) t_{2} v\right) \\
        &=f\left(\alpha\left(x+t_{1} v\right)+(1-\alpha)\left(x+t_{2} v\right)\right)
        \end{aligned}
  \end{equation*}

由于$f$是凸的,则有,
\begin{equation*}
    \begin{aligned}
    g\left(\alpha t_{1}+(1-\alpha) t_{2}\right) & \leq \alpha f\left(x+t_{1} v\right)+(1-\alpha) f\left(x+t_{2} v\right) \\
    &=\alpha g\left(t_{1}\right)+(1-\alpha) g\left(t_{2}\right)
    \end{aligned}
\end{equation*}

因此,$g$是凸的。\\

\textbf{再证:}$g$是凸的$\Longrightarrow$ $f$是凸的

取$\forall x_{1} ,x_{2} \in \mathbf{dom}\ f$,并且$\alpha  \in \left [ 0,1 \right ]$.

我们要证,\[f\left(\alpha x_{1}+(1-\alpha) x_{2}\right) \leq \alpha f\left(x_{1}\right)+(1-\alpha) f\left(x_{2}\right) \]

令$\lambda =x_{2}-x_{1}$,考虑$g\left(t\right)=f\left(x_{1}+t\lambda \right)=f\left(x_{1}+t\left(x_{2}-x_{1}\right)\right)$.

容易验证,$g\left(0\right)=f\left(x_{1}\right)$,$g\left(1\right)=f\left(x_{2}\right)$,$g\left(1-\alpha \right)=f\left(\alpha x_{1}+\left(1-\alpha \right)x_{2}\right)$.

则有,
\begin{equation*}
\begin{aligned}
    f(\alpha x_{1}+(1-\alpha) x_{2}) &=g(1-\alpha) \\
    &=g(\alpha \cdot 0+(1-\alpha) \cdot 1) \\
    & \leq \alpha g(0)+(1-\alpha) g(1) \\
    &=\alpha f(x_{1})+(1-\alpha) f(x_{2})
    \end{aligned}
\end{equation*}

因此,$f$是凸的。\\

原命题得证。

\hfill $\Box$ 

\end{document}

%%%%%%%%%%%%%%%%%%%%%%%%%%%%%%%%%%%%%%%%%%%%%%%%%%%%%%%%%%%%%%%%%%
%%%%%%%%%%%%%%%%%%%%%%%%%%%%%%%%%%%%%%%%%%%%%%%%%%%%%%%%%%%%%%%%%%
