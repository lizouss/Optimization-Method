%%%%%%%%%%%%%%%%%%%%%%%%%%%%%%%%%%%%%%%%%%%%%%%%%%%%%%%%%%%%%%%%%%
%%%%%%%%%%%%%%%%%%%%%%%%%%%%%%%%%%%%%%%%%%%%%%%%%%%%%%%%%%%%%%%%%%
%Packages
\documentclass[10pt, a4paper]{article}
\usepackage[UTF8]{ctex}
\usepackage[top=3cm, bottom=4cm, left=3.5cm, right=3.5cm]{geometry}
\usepackage{amsmath,amsthm,amsfonts,amssymb,amscd, fancyhdr, color, comment, graphicx, environ}
\usepackage{float}
\usepackage{mathrsfs}
\usepackage[math-style=ISO]{unicode-math}
\setmathfont{TeX Gyre Termes Math}
\usepackage{lastpage}
\usepackage[dvipsnames]{xcolor}
\usepackage[framemethod=TikZ]{mdframed}
\usepackage{enumerate}
\usepackage[shortlabels]{enumitem}
\usepackage{fancyhdr}
\usepackage{indentfirst}
\usepackage{listings}
\usepackage{sectsty}
\usepackage{thmtools}
\usepackage{shadethm}
\usepackage{hyperref}
\usepackage{CJKfntef}
\usepackage{setspace}
\hypersetup{
    colorlinks=true,
    linkcolor=blue,
    filecolor=magenta,      
    urlcolor=blue,
}
%%%%%%%%%%%%%%%%%%%%%%%%%%%%%%%%%%%%%%%%%%%%%%%%%%%%%%%%%%%%%%%%%%
%%%%%%%%%%%%%%%%%%%%%%%%%%%%%%%%%%%%%%%%%%%%%%%%%%%%%%%%%%%%%%%%%%
%Environment setup
\mdfsetup{skipabove=\topskip,skipbelow=\topskip}
\newrobustcmd\ExampleText{%
An \textit{inhomogeneous linear} differential equation has the form
\begin{align}
L[v ] = f,
\end{align}
where $L$ is a linear differential operator, $v$ is the dependent
variable, and $f$ is a given non−zero function of the independent
variables alone.
}
\mdfdefinestyle{theoremstyle}{%
linecolor=black,linewidth=1pt,%
frametitlerule=true,%
frametitlebackgroundcolor=gray!20,
innertopmargin=\topskip,
}
\mdtheorem[style=theoremstyle]{Problem}{作业题}
\newenvironment{Solution}{}

%%%%%%%%%%%%%%%%%%%%%%%%%%%%%%%%%%%%%%%%%%%%%%%%%%%%%%%%%%%%%%%%%%
%%%%%%%%%%%%%%%%%%%%%%%%%%%%%%%%%%%%%%%%%%%%%%%%%%%%%%%%%%%%%%%%%%
%Fill in the appropriate information below
\newcommand{\norm}[1]{\left\lVert#1\right\rVert}     
\newcommand\course{最优化方法}                         % <-- course name   
%\newcommand\hwnumber{1}                              % <-- homework number
\newcommand\Information{李邹/人工智能一班}          % <-- personal information
%%%%%%%%%%%%%%%%%%%%%%%%%%%%%%%%%%%%%%%%%%%%%%%%%%%%%%%%%%%%%%%%%%
%%%%%%%%%%%%%%%%%%%%%%%%%%%%%%%%%%%%%%%%%%%%%%%%%%%%%%%%%%%%%%%%%%
%Page setup
\pagestyle{fancy}
\headheight 35pt
\lhead{\today}
\rhead{\includegraphics[width=2.5cm]{lzu-logo.png}}
\lfoot{}
\pagenumbering{arabic}
\cfoot{\small\thepage}
\rfoot{}
\headsep 1.2em
\renewcommand{\baselinestretch}{1.25}
%%%%%%%%%%%%%%%%%%%%%%%%%%%%%%%%%%%%%%%%%%%%%%%%%%%%%%%%%%%%%%%%%%
%%%%%%%%%%%%%%%%%%%%%%%%%%%%%%%%%%%%%%%%%%%%%%%%%%%%%%%%%%%%%%%%%%
%Add new commands here
\renewcommand{\labelenumi}{\alph{enumi})}
\newcommand{\Z}{\mathbb Z}
\newcommand{\R}{\mathbb R}
\newcommand{\Q}{\mathbb Q}
\newcommand{\NN}{\mathbb N}
\DeclareMathOperator{\Mod}{Mod} 
\renewcommand\lstlistingname{Algorithm}
\renewcommand\lstlistlistingname{Algorithms}
\def\lstlistingautorefname{Alg.}
%%%%%%%%%%%%%%%%%%%%%%%%%%%%%%%%%%%%%%%%%%%%%%%%%%%%%%%%%%%%%%%%%%
%%%%%%%%%%%%%%%%%%%%%%%%%%%%%%%%%%%%%%%%%%%%%%%%%%%%%%%%%%%%%%%%%%
%Begin now!



\begin{document}

\begin{titlepage}
    \begin{center}
        \vspace*{3.5cm}
            
        \Huge
        \textbf{作业9}
            
        \vspace{2cm}
        \LARGE
        李邹
            
        \vspace{0.1cm}
        \Large
        人工智能一班(2020级)                      % <-- author
        
            
        \vfill
        
        \course \ 课程作业
            
        \vspace{1cm}
            
        \includegraphics[width=0.4\textwidth]{lzu-logo.png}
        \\
        
        \Large
        
        \today
            
    \end{center}
\end{titlepage}

%%%%%%%%%%%%%%%%%%%%%%%%%%%%%%%%%%%%%%%%%%%%%%%%%%%%%%%%%%%%%%%%%%
%%%%%%%%%%%%%%%%%%%%%%%%%%%%%%%%%%%%%%%%%%%%%%%%%%%%%%%%%%%%%%%%%%
%Start the assignment now

%%%%%%%%%%%%%%%%%%%%%%%%%%%%%%%%%%%%%%%%%%%%%%%%%%%%%%%%%%%%%%%%%%
%New problem
\newpage
\begin{Problem}
    证明 $x^{\star}=(1,1 / 2,-1)$ 是优化问题
    $$
    \begin{array}{ll}
    \operatorname{minimize} & (1 / 2) x^{T} P x+q^{T} x+r \\
    \text {subject to } & -1 \leq x_{i} \leq 1, \quad i=1,2,3,
    \end{array}
    $$
    的最优解,其中
    $$
    P=\left[\begin{array}{rrr}
    13 & 12 & -2 \\
    12 & 17 & 6 \\
    -2 & 6 & 12
    \end{array}\right], \quad q=\left[\begin{array}{r}
    -22.0 \\
    -14.5 \\
    13.0
    \end{array}\right], \quad r=1
    $$
\end{Problem}
    
\begin{Solution}
\textbf{证明}
\end{Solution}
%Complete the assignment now
依题意有
\[\nabla f_{0}(x)^{T}=P x+q, \quad \nabla f_{0}\left(x^{*}\right)^{T}=(-1,0,2)\]

所以有
\[\nabla f_{0}\left(x^{*}\right)^{T}\left(y-x^{*}\right)=-\left(y_{1}-1\right)+0\times \left(y_{2}-\frac{1}{2} \right)+2\times \left(y_{3}+1\right)\]

又因为$-1\le y_{i} \le 1$,所以对于全体$y_{i}$
\begin{align}
    \nabla f_{0}\left(x^{*}\right)^{T}\left(y-x^{*}\right) & = -\left(y_{1}-1\right)+2\left(y_{3}+1\right) \notag \\
    & = 2y_{3}-y_{1}+3 \geq 0 
\end{align}

(1)式大于等于0恒成立

所以$x^{\star}=(1,1 / 2,-1)$是该优化问题的最优解。
\hfill $\Box$
\vspace*{3em}

\begin{Problem}
\textbf{凸-凹分式问题。}\quad 考虑具有下面形式的问题
$$
\begin{array}{ll}
\operatorname{minimize} & f_{0}(x) /\left(c^{T} x+d\right) \\
\text {subject to } & f_{i}(x) \leq 0, \quad i=1, \ldots, m \\
& A x=b
\end{array}
$$
其中$f_{0}, f_{1}, \ldots, f_{m}$为凸,而目标函数的定义域为$\left\{x \in \operatorname{dom} f_{0} \mid c^{T} x+d>0\right\}$。
\begin{enumerate}[(a)]
    \item 证明这是一个拟凸优化问题。
    \item 证明这个问题等价于
    $$
    \begin{array}{ll}
    \text { minimize } & g_{0}(y, t) \\
    \text { subject to } & g_{i}(y, t) \leq 0, \quad i=1, \ldots, m \\
    & A y=b t \\
    & c^{T} y+d t=1
    \end{array}
    $$
    其中$g_{i}$是$f_{i}$的透视。其变量是$y \in \mathbf{R}^{n}$ 和 $t \in \mathbf{R}$。说明这个问题是凸的。
\end{enumerate}
\end{Problem}

\begin{Solution}
\textbf{证明}
\end{Solution}
%Complete the assignment now
\begin{enumerate}[(a)]
    \item 显然,该优化问题的不等式约束是凸的,等式约束是仿射的。因此问题转换为证明其目标函数是拟凸的。
    
    即证明$ f_{0}(x) /\left(c^{T} x+d\right)$是拟凸函数。

    该函数的下水平集为
    \[S_{\alpha}=\left\{x_{i} \in \operatorname{dom} g \mid g(x) \leq \alpha\right\}, \alpha \in R\]

    对于$\forall x, y \in\left\{x \in \operatorname{dom} f_{0} \mid c^{T} x+d>0\right\},\theta \in \left [ 0,1 \right ] $,有
    \begin{align}
        &c^{T}(\theta x+(1-\theta) y)+d \notag \\ 
        = &c^{T}(\theta x+(1-\theta y))+\theta d+(1-\theta) d>0 \notag
    \end{align}

    因此目标函数的定义域是凸集。

    不妨设
    \begin{align}
        g(x) & = f_{0}(x) /\left(c^{T} x+d\right) \leq \alpha \notag \\
        g(y) & = f_{0}(y) /\left(c^{T} y+d\right) \leq \alpha \notag
    \end{align}
    那么有
    \[g(\theta x+(1-\theta) y)=\frac{f_{0}(\theta x+(1-\theta) y)}{\left(c^{T}(\theta x+(1-\theta) y)+d\right)}\]

    因为已知$f_{0}$是凸函数,所以
    \[g(\theta x+(1-\theta) y) \leq \frac{\theta f_{0}(x)+(1-\theta) f_{0}(y)}{\left(c^{T}(\theta x+(1-\theta) y)+d\right)} \leq \alpha\]

    可知$g(x)$的任意下水平集均为凸集,所以$g(x)$为拟凸函数。

    \textbf{综上所述,该问题为拟凸优化问题。}
    \item 因为$g_{i}(y, t)$是$f_{i}(i)$的透视函数,所以有
    \[g_{i}(y, t)=t f_{i}(y / t), \quad y / t \in\left\{x \in d o m f_{0} \mid c^{T} x+d>0\right\}, t>0\]

    原问题的目标函数可转换为
    \[\frac{f_{i}(y / t)}{\left(c^{T} y / t+d\right)} \Leftrightarrow \frac{t f_{i}(y / t)}{\left(c^{T} y+t d\right)}\]
    
    由于存在约束$c^{T} y+d t=1$,因此两问题的目标函数等价。

    而由于
    \[g_{i}(y, t)=t f_{i}(y / t) \leq 0 \Rightarrow f_{i}(y / t) \leq 0\]

    因此两问题的不等式约束等价。

    又因为
    \[A y=b t \Leftrightarrow  A \frac{y}{t}=b \]

    因此等式约束亦是等价的,由等价性,结合(a)可知,该优化问题是凸的。

    \textbf{综上所述,两问题等价。}
\end{enumerate}
\hfill $\Box$
\end{document}

%%%%%%%%%%%%%%%%%%%%%%%%%%%%%%%%%%%%%%%%%%%%%%%%%%%%%%%%%%%%%%%%%%
%%%%%%%%%%%%%%%%%%%%%%%%%%%%%%%%%%%%%%%%%%%%%%%%%%%%%%%%%%%%%%%%%%
